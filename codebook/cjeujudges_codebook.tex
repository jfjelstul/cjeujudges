%--------------------------------------------------%
% generated by the codebookr R package
% created by Joshua C. Fjelstul, Ph.D.
%--------------------------------------------------%

\documentclass[10pt]{article}

%--------------------------------------------------%
% packages
%--------------------------------------------------%

% page layout
\usepackage{geometry}

% fonts
\usepackage[english]{babel}
\usepackage{underscore}
\usepackage{anyfontsize}
\usepackage[utf8]{inputenc}
\usepackage[T1]{fontenc}
\usepackage{fontspec}

% graphics and tables
\usepackage{graphicx} % add figures
\usepackage{xcolor} % change font color
\usepackage{tikz} % add graphics

% paragraph spacing
\usepackage{setspace}

% hyperlinks
\usepackage{url}

% table of contents
\usepackage{tocloft}

% test alignment
\usepackage{ragged2e}

% multi-page tables
\usepackage{longtable}

% custom lists
\usepackage{enumitem}

% insert content on every page
\usepackage{atbegshi} 

% code formatting
\usepackage{tcolorbox}

%--------------------------------------------------%
% colors
%--------------------------------------------------%

% define colors
\definecolor{themecolor}{HTML}{4B94E6}
\definecolor{background}{HTML}{EEF6FD}

% format hyperlinks
\usepackage[colorlinks=true,linkcolor=themecolor,citecolor=themecolor,urlcolor=themecolor,breaklinks=true]{hyperref}

%--------------------------------------------------%
% formatting
%--------------------------------------------------%

% configure main font
\setmainfont[Ligatures=TeX,BoldFont={Roboto Medium}]{Roboto Light}
\setmonofont[Ligatures=TeX]{Roboto Mono-Light}

% set page margins
\geometry{top = 1.5in, bottom = 1.5in, left = 1.5in, right = 1.5in}

% set paper size
\geometry{letterpaper}

% format table of contents
\renewcommand{\cftsecdotsep}{10}
\renewcommand{\cftsecleader}{\cftdotfill{\cftdotsep}}
\renewcommand{\cftsecfont}{{\small\color{black!75}\bfseries}}
\renewcommand{\cftsecpagefont}{{\small\color{black!75}\normalfont}}

% adjust spacing
\usepackage{parskip}
\parskip=10pt
\renewcommand{\baselinestretch}{1.4}

% hyphen formatting
\hyphenpenalty = 10000
\exhyphenpenalty = 10000

% prevent widow and orphan lines
\widowpenalty10000
\clubpenalty10000

%--------------------------------------------------%
% page elements
%--------------------------------------------------%

% a command to make a code box
\newtcbox{\codebox}{nobeforeafter,tcbox raise base,colback=black!5,colframe=white,coltext=black!75,boxrule=0pt,arc=3pt,boxsep=0pt,
left=4pt,right=4pt,top=3pt,bottom=3pt}

% a command to make a chip
\newtcbox{\chip}{nobeforeafter,tcbox raise base,colback=black!5,colframe=white,coltext=black!75,boxrule=0pt,arc=11pt,boxsep=0pt,
left=10pt,right=10pt,top=8pt,bottom=8pt}

% command to format code
\newcommand{\code}[1]{\codebox{{\footnotesize\texttt{#1}}}}

% command to highlight text
\newcommand{\highlight}[1]{{\color{themecolor} \textbf{#1}}}

% command to create a divider
\newcommand{\dividerline}{{\color{gray!10} \rule[4pt] {\textwidth}{3pt}}}

% command to add a cover
\newcommand{\cover}[4]{
\begin{tikzpicture}[remember picture,overlay, shift={(current page.south west)}]
\fill[themecolor] (0, 5.5in) rectangle ++ (8.5in, 5.5in); % header bar
\fill[black!5] (0, 4in) rectangle ++ (8.5in, 1.5in); % middle bar
\fill[white] (0, 0in) rectangle ++ (8.5in, 4in); % footer bar
\node[anchor=west] at (1.5in, 6.25in) {\color{white} \fontsize{60}{60}\selectfont \begin{minipage}{5.5in} \textbf{Codebook} \fontsize{15}{15}\selectfont \hspace{5pt} v #2 \end{minipage}};
\node[anchor=west, align=left] at (1.5in, 4.75in) {\begin{minipage}{5.5in} \color{black!40} \fontsize{#4}{#4} \selectfont #1 \end{minipage}};
\node[anchor=west, align=left, minimum height=2in] at (1.5in, 2.55in) {\begin{minipage}[t][2in]{5.5in} \color{black!40} \fontsize{10}{10} \selectfont #3 \end{minipage}};
\end{tikzpicture}
}

% command to add a header page
\newcommand{\headerpage}[4]{
	\newpage
	\begin{tikzpicture}[remember picture,overlay, shift={(current page.south west)}]
		\fill[themecolor] (0, 9in) rectangle ++ (8.5in, 2in); % header line 1
		\fill[black!5] (0, 8in) rectangle ++ (8.5in, 1in); % header line 2
		\node[anchor = west] at (1.5in, 9.6in) {\color{white} \fontsize{#3}{#3}\selectfont \textbf{#1}}; % heading
		\node[anchor = west] at (1.5in, 8.5in) {\color{black!40} \fontsize{#4}{#4}\selectfont #2}; % heading
	\end{tikzpicture}
	\phantomsection
	\addcontentsline{toc}{section}{#1}
	\vspace{1.5in}
}

% command to layout page
\newcommand\pagelayout{
	\begin{tikzpicture}[remember picture,overlay, shift={(current page.south west)}]
		% \fill[themecolor] (0, 10.75in) rectangle ++ (8.5in, 0.25in); % header
		\fill[black!5] (0, 0) rectangle ++ (8.5in, 0.5in); % footer
		\draw (0.25in, 0.25in) node[anchor = west] {\fontsize{9}{9}\selectfont \color{black!40} The CJEU Judges Database Codebook \hspace{5pt} | \hspace{5pt} Joshua C. Fjelstul, Ph.D.}; % footer content
		\draw (8.25in, 0.25in) node[anchor = east] {\fontsize{9}{9}\selectfont \color{black!40} \thepage}; % page number
	\end{tikzpicture}
}

% add page layout 
\AtBeginShipout{
	\AtBeginShipoutUpperLeft{\pagelayout}
}

% command to add a subheading
\newcommand{\subheading}[1]{
\vspace{24pt}
{\color{themecolor} \fontsize{14}{14}\selectfont \textbf{#1}}
\vspace{6pt}
\dividerline
\vspace{-20pt}
}

%--------------------------------------------------%
% start document
%--------------------------------------------------%

\begin{document}

\clearpage
\pagestyle{empty}

\color{black!75}

\small

\begin{flushleft}

%--------------------------------------------------%
% cover
%--------------------------------------------------%

\cover{The CJEU Judges Database}{1.0}{Joshua C. Fjelstul, Ph.D.}{16}

\newpage

%--------------------------------------------------%
% table of contents
%--------------------------------------------------%

% reset page counter
\setcounter{page}{1}

% format the table of contents header
% \renewcommand\contentsname{{\color{themecolor} \fontsize{14}{14}\selectfont Datasets}}
\renewcommand\contentsname{\subheading{Datasets} \vspace{0pt}}

% add the table of contents
\tableofcontents

% remove page number from table of contents pages
\addtocontents{toc}{\protect\thispagestyle{empty}}

\newpage

%--------------------------------------------------%
% content
%--------------------------------------------------%


%--------------------------------------------------%
% dataset
%--------------------------------------------------%

\headerpage{judges}{Data on CJEU judges}{24}{10}

\subheading{Description}

This dataset includes data on all judges, Advocate General, and registrars at the Court of Justice (1952-2022), the General Court (1989-2022), and the Civil Service Tribunal (2005-2016). There is one observation per individual who has served on the Court. The dataset includes each individual's first and last name, their last name using only ASCII characters (to avoid character-encoding problems), a label for each individual (for making visualizations) that differentiates between judges with the same last name, their member state, and their gender. The dataset also indicates the positions that each indvidual has held at the Court and the start date and end date for each position.

\subheading{Variables}

\begin{description}[labelwidth=130pt, leftmargin=\dimexpr\labelwidth+\labelsep\relax, font=\normalfont, itemsep=10pt]
\item[\code{key\_id}] \code{numeric}\hspace{5pt}An ID number that uniquely identifies each observation. Indicates the default sort order for the dataset.
\item[\code{iuropa\_judge\_id}] \code{string}\hspace{5pt}An ID number that uniquely identifies each individual in the format \code{J:\#\#\#\#}. The first two digits are the member state ID, with a leading 0 if necessary. The second two digits uniquely identify the individual within their member state. The number is assigned with the individuals sorted chronologically by date of appointment and then alphabetically by last name. 
\item[\code{full\_name}] \code{string}\hspace{5pt}The full name of the individual. 
\item[\code{first\_name}] \code{string}\hspace{5pt}The first name of the individual. 
\item[\code{last\_name}] \code{string}\hspace{5pt}The last name of the individual.
\item[\code{last\_name\_latin}] \code{string}\hspace{5pt}The last name of the individual using only basic Latin characters (to avoid character-encoding problems).
\item[\code{last\_name\_label}] \code{string}\hspace{5pt}A label (for making visualizations) that differentiates between individual with the same last name by adding the individual's first initial.
\item[\code{last\_name\_latin\_label}] \code{string}\hspace{5pt}A label (for making visualizations) using only basic Latin characters (to avoid character-encoding problems) that differentiates between individual with the same last name by adding the individual's first initial.
\item[\code{member\_state\_id}] \code{numeric}\hspace{5pt}An ID number that uniquely identifies each individual's member state. The number is assigned with member states sorted by date of accession and then alphabetically by last name. 
\item[\code{member\_state}] \code{string}\hspace{5pt}The name of the individual's member state. 
\item[\code{member\_state\_code}] \code{string}\hspace{5pt}A two-character code for the individual's member state assigned by the Commission.
\item[\code{birth\_year}] \code{numeric}\hspace{5pt}The year the individual was born.
\item[\code{gender\_id}] \code{string}\hspace{5pt}An ID variable indicating the gender of the individual. Coded 1 if the judge is female and 2 if the individual is male. 
\item[\code{gender}] \code{string}\hspace{5pt}The gender of the individual. Either \code{Male} or \code{Female}.
\item[\code{female}] \code{string}\hspace{5pt}A dummy variable indicating whether the individual is female. Coded 1 if the individual is female and 0 if the individual is male. 
\item[\code{judge\_court\_of\_justice}] \code{dummy}\hspace{5pt}A dummy variable indicating whether the individual served as a judge at the Court of Justice. Coded 1 if the judge was a individual at the Court of Justice and 0 otherwise.
\item[\code{judge\_general\_court}] \code{dummy}\hspace{5pt}A dummy variable indicating whether the individual served as a judge at the General Court. Coded 1 if the judge was a individual at the General Court and 0 otherwise.
\item[\code{judge\_civil\_service\_tribunal}] \code{dummy}\hspace{5pt}A dummy variable indicating whether the individual served as a judge at the Civil Service Tribunal. Coded 1 if the individual was a judge at the Civil Service Tribunal and 0 otherwise.
\item[\code{advocate\_general}] \code{dummy}\hspace{5pt}A dummy variable indicating whether the individual served as an Advocate General at the Court of Justice. Coded 1 if the individual was an Advocate General at the Court of Justice and 0 otherwise.
\item[\code{registrar\_court\_of\_justice}] \code{dummy}\hspace{5pt}A dummy variable indicating whether the individual served as the registrar at the Court of Justice. Coded 1 if the individual was the registrar at the Court of Justice and 0 otherwise.
\item[\code{registrar\_general\_court}] \code{dummy}\hspace{5pt}A dummy variable indicating whether the individual served as the registrar at the General Court. Coded 1 if the individual was the registrar at the General Court and 0 otherwise.
\item[\code{registrar\_civil\_service\_tribunal}] \code{dummy}\hspace{5pt}A dummy variable indicating whether the individual served as the registrar at the Civil Service Tribunal. Coded 1 if the individual was the registrar at the Civil Service Tribunal and 0 otherwise.
\item[\code{current\_status}] \code{string}\hspace{5pt}A description that indicates the current status of the judge on the Court (e.g., current CJ judge, former GC judge). 
\item[\code{current\_member}] \code{dummy}\hspace{5pt}A dummy variable indicating whether the judge is currently a member of the Court.
\item[\code{count\_positions}] \code{numeric}\hspace{5pt}The number of positions that the individual has held at the Court. Possible positions include judge or registrar at the Court of Justice, General Court, or Civil Service Tribunal, and Advocate General at the Court of Justice. 
\item[\code{nonconsecutive\_positions}] \code{dummy}\hspace{5pt}A dummy variable indicating whether the individual served non-consecutive terms in the same position. 
\item[\code{start\_date}] \code{date}\hspace{5pt}The start date of the individual's first position at the Court.
\item[\code{end\_date}] \code{date}\hspace{5pt}The end date of the individual's last position at the Court.
\item[\code{start\_date\_judge\_court\_of\_justice}] \code{date}\hspace{5pt}The start date of the individual's tenure as a judge at the Court of Justice, if applicable. Coded \code{NA} if not applicable. 
\item[\code{end\_date\_judge\_court\_of\_justice}] \code{date}\hspace{5pt}The end date of the individual's tenure as a judge at the Court of Justice, if applicable. Coded \code{NA} if not applicable. If a judge served non-consecutive terms, this is the end date of the last term. 
\item[\code{start\_date\_judge\_general\_court}] \code{date}\hspace{5pt}The start date of the individual's tenure as a judge at the General Court, if applicable. Coded \code{NA} if not applicable. 
\item[\code{end\_date\_judge\_general\_court}] \code{date}\hspace{5pt}The end date of the individual's tenure as a judge at the General Court, if applicable. Coded \code{NA} if not applicable. If a judge served non-consecutive terms, this is the end date of the last term. 
\item[\code{start\_date\_judge\_civil\_service\_tribunal}] \code{date}\hspace{5pt}The start date of the individual's tenure as a judge at the Civil Service Tribunal, if applicable. Coded \code{NA} if not applicable. 
\item[\code{end\_date\_judge\_civil\_service\_tribunal}] \code{date}\hspace{5pt}The end date of the individual's tenure as a judge at the Civil Service Tribunal, if applicable. Coded \code{NA} if not applicable. If a judge served non-consecutive terms, this is the end date of the last term. 
\item[\code{start\_date\_advocate\_general}] \code{date}\hspace{5pt}The start date of the individual's tenure as an Advocate General at the Court of Justice, if applicable. Coded \code{NA} if not applicable. 
\item[\code{end\_date\_advocate\_general}] \code{date}\hspace{5pt}The end date of the individual's tenure as an Advocate General at the Court of Justice, if applicable. Coded \code{NA} if not applicable. If a judge served non-consecutive terms, this is the end date of the last term. 
\end{description}
%--------------------------------------------------%
% dataset
%--------------------------------------------------%

\headerpage{judge\_backgrounds}{Data on the backgrounds of CJEU judges}{24}{10}

\subheading{Description}

This dataset includes data on professional backgrounds of all judges at the Court of Justice (1953-2021) and the General Court (1989-2021). The sources of the data are the judges' official biographies, which are published online by the Court. There is one observation per biographical item per judge. The dataset indicates whether each item relates to the judge's education, the judge's prior professional experience, or some other category of biographical information (such as membership in a professional society or professional accomplishments). For items related to a judge's professional experience, the dataset indicates whether the judge was a judge (at a different court), a lawyer, a civil servant, an academic, or a politician. For items related to prior experience as a judge, the dataset indicates whether the court was a lower court, a high court, the CJEU, or another international court.

\subheading{Variables}

\begin{description}[labelwidth=130pt, leftmargin=\dimexpr\labelwidth+\labelsep\relax, font=\normalfont, itemsep=10pt]
\item[\code{key\_id}] \code{numeric}\hspace{5pt}An ID number that uniquely identifies each observation. Indicates the default sort order for the dataset.
\item[\code{iuropa\_judge\_id}] \code{string}\hspace{5pt}An ID number that uniquely identifies each judge in the format \code{J:\#\#\#\#}. The first two digits are the member state ID, with a leading 0 if necessary. The second two digits uniquely identify the judge within the member state. The number is assigned with the judges sorted chronologically by date of appointment and then alphabetically by last name. 
\item[\code{item\_id}] \code{string}\hspace{5pt}An ID number that uniquely identifies each biography item in the format \code{J:\#\#\#\#:\#\#}. This number is the judge ID followed by the number of the item, with a leading 0 if necessary.
\item[\code{first\_name}] \code{string}\hspace{5pt}The first name of the judge. 
\item[\code{last\_name}] \code{string}\hspace{5pt}The last name of the judge.
\item[\code{last\_name\_latin}] \code{string}\hspace{5pt}The last name of the judge using only basic Latin characters (to avoid character-encoding problems).
\item[\code{last\_name\_label}] \code{string}\hspace{5pt}A label (for making visualizations) that differentiates between judges with the same last name by adding the judge's first initial.
\item[\code{last\_name\_latin\_label}] \code{string}\hspace{5pt}A label (for making visualizations) using only basic Latin characters (to avoid character-encoding problems) that differentiates between judges with the same last name by adding the judge's first initial.
\item[\code{item\_number}] \code{numeric}\hspace{5pt}The number of the item in the judge's biography.
\item[\code{item\_text}] \code{string}\hspace{5pt}The text of the bibliography item.
\item[\code{type\_education}] \code{dummy}\hspace{5pt}A dummy variable indicating whether the item relates to the judge's education.
\item[\code{type\_job}] \code{dummy}\hspace{5pt}A dummy variable indicating whether the item relates to the judge's prior professional experience.
\item[\code{type\_other}] \code{dummy}\hspace{5pt}A dummy variable indicating whether the item relates to some other biographical information, such as membership in a professional organization or a professional accomplishment. 
\item[\code{job\_judge}] \code{dummy}\hspace{5pt}A dummy variable indicating whether the item says that the judge has prior experience as a lawyer. Coded \code{0} if \code{type\_job} is not coded \code{1}.
\item[\code{job\_lawyer}] \code{dummy}\hspace{5pt}A dummy variable indicating whether the item says that the judge has prior experience as a judge. Coded \code{0} if \code{type\_job} is not coded \code{1}.
\item[\code{job\_civil\_servant}] \code{dummy}\hspace{5pt}A dummy variable indicating whether the item says that the judge has prior experience as a civil servant. Coded \code{0} if \code{type\_job} is not coded \code{1}.
\item[\code{job\_academic}] \code{dummy}\hspace{5pt}A dummy variable indicating whether the item says that the judge has prior experience as an academic. Coded \code{0} if \code{type\_job} is not coded \code{1}.
\item[\code{job\_politician}] \code{dummy}\hspace{5pt}A dummy variable indicating whether the item says that the judge has prior experience as a politician. Coded \code{0} if \code{type\_job} is not coded \code{1}.
\item[\code{court\_lower}] \code{dummy}\hspace{5pt}A dummy variable indicating whether the item says that the judge has prior experience as a judge at a lower court. Coded \code{0} if \code{job\_judge} is not coded \code{1}.
\item[\code{court\_high}] \code{dummy}\hspace{5pt}A dummy variable indicating whether the item says that the judge has prior experience as a judge at a high court. Coded \code{0} if \code{job\_judge} is not coded \code{1}.
\item[\code{court\_cjeu}] \code{dummy}\hspace{5pt}A dummy variable indicating whether the item says that the judge has prior experience as a judge or an Advocate General at the CJEU. Coded \code{0} if \code{job\_judge} is not coded \code{1}.
\item[\code{court\_international}] \code{dummy}\hspace{5pt}A dummy variable indicating whether the item says that the judge has prior experience as a judge at an international court (excluding the CJEU). Coded \code{0} if \code{job\_judge} is not coded \code{1}.
\item[\code{common\_law\_university}] \code{dummy}\hspace{5pt}A dummy variable indicating whether the item mentions a common law university.
\end{description}

%--------------------------------------------------%
% end document
%--------------------------------------------------%

\end{flushleft}

\end{document}
